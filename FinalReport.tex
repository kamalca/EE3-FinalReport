\documentclass[12pt]{article}
%%%%%%%%%%%%%%%%%%%%%%%%%%%%%%%%%%%%%%%%%%
\usepackage{amsmath}
\usepackage[margin=1in]{geometry}
\usepackage{tabu}
\usepackage{makecell}
\usepackage[siunitx, american]{circuitikz}

\usepackage{tikz}
\usepackage{pgfplots}
\pgfplotsset{compat=newest}

\newcommand{\ohm}{$\Omega$}
\tabulinesep=1.2mm
%%%%%%%%%%%%%%%%%%%%%%%%%%%%%%%%%%%%%%%%%%

\title{ECE 3 Final Report}
\author{Kameron Carr \\ 504988167 \and Hien Ho \\ 505190709}
\date{TA: Xin Li \\ Section 1A \\ June 15, 2019}

\begin{document}
\maketitle

%1
\section{Introduction and Background}

%2
\section{Testing Methodology}
%%2.1
\subsection{Motors}
%%%2.1.1
\subsubsection{Test Setup}
To start, you may need to assemble your car. This process includes assembling the wheels, putting the motors into the clamps on the car, and putting the LaunchPad on the car. You may need a TA's help to solder the header connections. Also it is important to remove the 5V jumper on the LaunchPad. If you are using the same car as the one used in this lab, not removing the jumper \textbf{may cause permanent damage to the car when plugging into a computer}.
\\
In order to be able control the motors in Energia, we first looked up the pin numbers on the MSP432 Pin Chart. From this chart we were able to find all the relevant pins to control the motors.
\\ \\
%%%Pin Chart
\begin{tabu}{|c|c|c|}
\hline
\textbf{Pin Functionality} & \textbf{Pin Number for Left Motor} & \textbf{Pin Number for Right Motor} \\ \hline
No Sleep (nSLP) & 31 & 11 \\
Direction (DIR) & 29 & 30 \\
Duty Cycle (PWM) & 40 & 39 \\
\hline
\end{tabu}
\\ \\ \\
Some of these pins also needed to be set to the correct mode and values. In Energia, the modes are set with the \texttt{pinMode} function, and the pin value is set through the \texttt{digitalWrite} function.
\\ \\
%%%Pin Values
\begin{tabu}{|c|c|c|}
\hline
\textbf{Pin Functionality} & \textbf{Pin Mode} & \textbf{Pin Value} \\ \hline
No Sleep (nSLP) & OUTPUT & HIGH \\
Direction (DIR) & OUTPUT & HIGH \\
\hline
\end{tabu} \\ \\
%%%2.1.2
\subsubsection{How the Tests were Conducted}
Knowing that the power to the motors ranges from 0-255, we wanted to get a feel for how fast the car went at different power levels. These are only rough guidelines of what we found to be true for our car. Note that these speeds can fluctuate depending on your car and the charge of your batteries. If you want to control the speed of the car with more precision, you must use the motor shaft encoders to measure the rotations.
\\ \\
%%%Motor Speeds
\begin{tabu}{|c|c|}
\hline
\textbf{Car Speed} & \textbf{Motor Power} \\ \hline
Not Moving &  0 - 20 \\
Slow       & 20 - 40 \\
Medium     & 40 - 90 \\
Fast       & 90 - 255 \\ \hline
\end{tabu}
\\ \\
Next we tested the evenness / straightness of the motors. We wanted to ensure that the car drove as straight as possible at the base power we would use in our path following program. Knowing the length of the course and the time limit, we were able to estimate that for our car to go fast enough, the base power of the motors would have to be around 60 on each motor. This is the target power we used when testing straightness. We noticed that going from zero power immediately to the target power would cause the car to jerk very quickly. For this reason, we used a for loop that changed the power of the motor incrementally with a small delay. This allowed the car to accelerate slowly and evenly without jerking around. We found this gave the most accurate results when trying to determine what power on each motor was needed to get the car to go straight.
\\ \\
%%%Motor Straightness
\begin{tabu}{|c|c|c|}
\hline
\textbf{Left Motor Power} & \textbf{Right Motor Power} & \textbf{Result} \\ \hline
62 & 60 & Pulled Right \\
60 & 60 & Straight \\
60 & 62 & Pulled Left \\ \hline
\end{tabu}

%%%2.1.3
\subsubsection{Data Analysis}
From these tests we found that the motors must have power of at least 20 to move consistently. A base power above 90 was found to be quite fast. We suspect a speed above this would make it significantly more difficult to respond to input from the path sensors and make appropriate adjustments.
\\ \\ 
We also found that our car goes straight when the power of each motor is set to 60. This suggests that the car will go straight unless the power of the motors is different.

%%%2.1.4
\subsubsection{Test Data Interpretation}
From these tests, we generally deduced that the motors were functioning as expected. By observing the general speeds of the motors at different power levels, we determined that a base power of 60 or greater would be a good starting goal for our car to finish the track in time. Also when using the shaft encoders to determine the speed of the vehicle, it is important to not go below 20 power on the motors, otherwise the feedback loop breaks because the motors are not turning. If the motors are not turning, the shaft encoders will not return information that can be used to calculated the speed.
\\ \\ 
The tests on motor straightness showed that our motors went at the same speed when the powers were equal. Therefore we did not include a bias value in our final code.

%%2.2
\subsection{Path Sensors}
%%%2.2.1
\subsubsection{Test Setup}

%%%2.2.2
\subsubsection{How the Tests were Conducted}

%%%2.2.3
\subsubsection{Data Analysis}

%%%2.2.4
\subsubsection{Test Data Interpretation}


%3
\section{Results and Discussion}

%4
\section{Conclusions and Future Work}

%5
\section{Illustration Credits}

%6
\section{References}


\end{document}
