\documentclass[12pt]{article}
%%%%%%%%%%%%%%%%%%%%%%%%%%%%%%%%%%%%%%%%%%
\usepackage{amsmath}
\usepackage[margin=1in]{geometry}
\usepackage{tabu}
\usepackage{makecell}
\usepackage[siunitx, american]{circuitikz}

\usepackage{tikz}
\usepackage{pgfplots}
\pgfplotsset{compat=newest}

\newcommand{\ohm}{$\Omega$}
\tabulinesep=1.2mm
%%%%%%%%%%%%%%%%%%%%%%%%%%%%%%%%%%%%%%%%%%

\title{ECE 3 Final Report}
\author{Kameron Carr \\ 504988167 \and Hien Ho \\ 505190709}
\date{TA: Xin Li \\ Section 1A \\ June 15, 2019}

\begin{document}
\maketitle

%1
\section{Introduction and Background}

%2
\section{Testing Methodology}
%%2.1
\subsection{Motors}
%%%2.1.1
\subsubsection{Test Setup}
To start, you may need to assemble your car. This process includes assembling the wheels, putting the motors into the clamps on the car, and putting the LaunchPad on the car. You may need a TA's help to solder the header connections. Also it is important to remove the 5V jumper on the LaunchPad. If you are using the same car as the one used in this lab, not removing the jumper \textbf{may cause permanent damage to the car when plugging into a computer}. Also make sure to put fresh batteries in your car and turn it on using the switch on the back end.
\\
In order to be able control the motors in Energia, we first looked up the pin numbers on the MSP432 Pin Chart. From this chart we were able to find all the relevant pins to control the motors.
\\ \\
%%%Pin Chart
\begin{tabu}{|c|c|c|}
\hline
\textbf{Pin Functionality} & \textbf{Pin Number for Left Motor} & \textbf{Pin Number for Right Motor} \\ \hline
No Sleep (nSLP) & 31 & 11 \\
Direction (DIR) & 29 & 30 \\
Duty Cycle (PWM) & 40 & 39 \\
\hline
\end{tabu}
\\ \\ \\
Some of these pins also needed to be set to the correct mode and values. In Energia, the modes are set with the \texttt{pinMode} function, and the pin value is set through the \texttt{digitalWrite} function.
\\ \\
%%%Pin Values
\begin{tabu}{|c|c|c|}
\hline
\textbf{Pin Functionality} & \textbf{Pin Mode} & \textbf{Pin Value} \\ \hline
No Sleep (nSLP) & OUTPUT & HIGH \\
Direction (DIR) & OUTPUT & HIGH \\
\hline
\end{tabu} \\ \\
%%%2.1.2
\subsubsection{How the Tests were Conducted}
Knowing that the power to the motors ranges from 0-255, we wanted to get a feel for how fast the car went at different power levels. These are only rough guidelines of what we found to be true for our car. Note that these speeds can fluctuate depending on your car and the charge of your batteries. If you want to control the speed of the car with more precision, you must use the motor shaft encoders to measure the rotations.
\\ \\
%%%Motor Speeds
\begin{tabu}{|c|c|}
\hline
\textbf{Car Speed} & \textbf{Motor Power} \\ \hline
Not Moving &  0 - 20 \\
Slow       & 20 - 40 \\
Medium     & 40 - 90 \\
Fast       & 90 - 255 \\ \hline
\end{tabu}
\\ \\
Next we tested the evenness / straightness of the motors. We wanted to ensure that the car drove as straight as possible at the base power we would use in our path following program. Knowing the length of the course and the time limit, we were able to estimate that for our car to go fast enough, the base power of the motors would have to be around 60 on each motor. This is the target power we used when testing straightness. We noticed that going from zero power immediately to the target power would cause the car to jerk very quickly. For this reason, we used a for loop that changed the power of the motor incrementally with a small delay. This allowed the car to accelerate slowly and evenly without jerking around. We found this gave the most accurate results when trying to determine what power on each motor was needed to get the car to go straight.
\\ \\
%%%Motor Straightness
\begin{tabu}{|c|c|c|}
\hline
\textbf{Left Motor Power} & \textbf{Right Motor Power} & \textbf{Result} \\ \hline
62 & 60 & Pulled Right \\
60 & 60 & Straight \\
60 & 62 & Pulled Left \\ \hline
\end{tabu}

%%%2.1.3
\subsubsection{Data Analysis}
From these tests we found that the motors must have power of at least 20 to move consistently. A base power above 90 was found to be quite fast. We suspect a speed above this would make it significantly more difficult to respond to input from the path sensors and make appropriate adjustments.
\\ \\ 
We also found that our car goes straight when the power of each motor is set to 60. This suggests that the car will go straight unless the power of the motors is different.

%%%2.1.4
\subsubsection{Test Data Interpretation}
From these tests, we generally deduced that the motors were functioning as expected. By observing the general speeds of the motors at different power levels, we determined that a base power of 60 or greater would be a good starting goal for our car to finish the track in time. Also when using the shaft encoders to determine the speed of the vehicle, it is important to not go below 20 power on the motors, otherwise the feedback loop breaks because the motors are not turning. If the motors are not turning, the shaft encoders will not return information that can be used to calculated the speed.
\\ \\ 
The tests on motor straightness showed that our motors went at the same speed when the powers were equal. Therefore we did not include a bias value in our final code.

%%2.2
\subsection{Path Sensors}
%%%2.2.1
\subsubsection{Test Setup}
As with the motors, testing the path sensors requires the car to have the batteries put in and the switch turned on.
\\ 
In order to see the values from the sensors, we will need to be able to print to the serial monitor. The serial monitor must be initialized using \texttt{Serial.begin} and a data rate of 9600 bps.
\\ 
To get readings from the sensors, we need to initialize the pins for the sensors as well as the infrared LEDs. We look at the pin chart to find the pin values we can use in Energia to refer to the LEDs and sensors.
\\ \\ 
%Sensor Pins
\begin{tabu}{|l|c|}
\hline
\makecell{\textbf{Device}} & \textbf{Pin Number} \\ \hline
Sensor 0 (Right Most) & 65 \\
Sensor 1              & 48 \\
Sensor 2              & 64 \\
Sensor 3              & 47 \\
Sensor 4              & 52 \\
Sensor 5              & 68 \\
Sensor 6              & 53 \\
Sensor 7 (Left Most)  & 69 \\
IR LEDs (Odd)         & 45 \\
IR LEDs (Even)        & 61 \\ \hline
\end{tabu}
\\ \\ \\ 
All of the LEDs are set to \texttt{OUTPUT} using the \texttt{pinMode} function and to \texttt{HIGH} using the \texttt{digitalWrite} function.

%%%2.2.2
\subsubsection{How the Tests were Conducted}
To get readings from the sensors, we created a for loop that goes through the 8 sensors. For each sensor, you first have to set it to \texttt{OUTPUT} and \texttt{HIGH} then delay for 10 ms to charge the capacitor in the sensor. Then you set the sensor to \texttt{INPUT} and delay for a set amount of time before getting a reading using the \texttt{digitalRead} function. We tested a few values for this second delay. We did this by running the program and printing the sensor values to the serial monitor with the \texttt{Serial.print} function. We then moved the car side to side over a paper with a black line printed on it.
\\ \\ 
%Sensor Delays
\begin{tabu}{|c|c|}
\hline
\textbf{Delay} & \textbf{Behavior} \\ \hline
4 ms & All Zeros \\ 
3 ms & Some Incorrect Zeros \\ 
2 ms & Correct \\ 
1 ms & Correct \\ 
\hline
\end{tabu}

%%%2.2.3
\subsubsection{Data Analysis}
The car properly identified the difference between the white paper and the black printed paper when the delay after turning a sensor to input was 1 or 2 ms. A delay greater than 2ms resulted in unexpected / incorrect results. The errors were when the value should have been 1 but was instead 0. This is expected as if the delay is too large, then the capacitor will have too much time to discharge.

%%%2.2.4
\subsubsection{Test Data Interpretation}
Originally we took these results and decided that 2ms would be a good delay to have for our path following program. However, later when we tried implementing the sensing of the horizontal black line for the purpose of turning around or stopping, we found that a delay of 1 ms resulted in more accurate results.

%3
\section{Results and Discussion}

%4
\section{Conclusions and Future Work}

%5
\section{Illustration Credits}

%6
\section{References}


\end{document}
